\documentclass[paper=a4, fontsize=12pt]{article}

\usepackage[portuguese]{babel}
\usepackage{amsmath,amsfonts,amssymb,amsthm}
\usepackage[utf8]{inputenc}
\usepackage{color}
\usepackage{proof}

\usepackage{iftex}
\pagestyle{empty}

\ifTUTeX
  \usepackage{fontspec}
\else
  \usepackage[T1]{fontenc}
  \usepackage[utf8]{inputenc} % The default since 2018
  \DeclareUnicodeCharacter{200B}{{\hskip 0pt}}
\fi

\newtheorem{theorem}{Teorema}
\newtheorem{lemma}{Lema}
\newtheorem{corollary}{Corolário}
\theoremstyle{definition}
\newtheorem{Example}{Exemplo}
\newtheorem{Definition}{Definição}


\begin{document}

\begin{center}
  UNIVERSIDADE FEDERAL DE OURO PRETO \\
  DEPARTAMENTO DE COMPUTAÇÃO\\
\end{center}
\vspace{2cm}

\begin{center}
{\bf\Large Título:} \Large{Semântica formal e validação de filtros de pacotes linux expressos na linguagem eBPF}
\end{center}
\vspace{4cm}
\begin{flushleft}
Aluno: Rafael Diniz de Oliveira\\
Orientador: Dr. Rodrigo Geraldo Ribeiro\\
\end{flushleft}
\vspace{4cm}
\noindent
Relatório Final, referente ao período 08/2020 a 07/2021, apresentado à
Universidade Federal de Ouro Preto, como parte das exigências do
programa de iniciação científica do edital EDITAL
\vspace{1cm}
\begin{center}
Ouro Preto - Minas Gerais - Brasil\\
\today
\end{center}

\clearpage

\begin{center}
{\bf\Large Resumo:}\\ \Large{Semântica formal e validação de filtros de pacotes linux expressos na linguagem eBPF}
\end{center}

\vspace{1cm}

O Extended Berkeley Packet Filter (eBPF) é um subsistema Linux que permite executar
extensões não confiáveis definidas pelo usuário dentro do kernel. Para proteger o
kernel contra código malicioso, o eBPF utiliza técnicas de análise estática simples.
Porém, a medida que a linguagem eBPF cresce em popularidade, o seu ecossistema evolui
para oferecer suporte a extensões mais complexas e diverisificadas. Porém, as limitações
presentes em seu verificador atual, como a sua alta taxa de falsos positivos e falta
de suporte a comandos de repetição, tornaram-se um grande empecilho para sua ampla
adoção por desenvolvedores de aplicações de rede. Neste sentido, o presente projeto
pretende aplicar técnicas de semântica formal e sistemas de tipos de maneira a
aumentar a suporte do verificador eBPF, permitindo assim que programadores
desenvolvam aplicações de maneira simples e garantindo a segurança do kernel
estendido.



\begin{center}
https://github.com/lives-group/racket-ebpf
\end{center}
\vspace{2cm}
\begin{center}
\begin{tabular}{c}
$\,$  \\
$\,$  \\
\hline
Bolsista: Rafael Diniz de Oliveira\\
$\,$  \\
$\,$  \\
$\,$  \\ \hline
Orientador: Prof. Dr. Rodrigo Geraldo Ribeiro
\end{tabular}
\end{center}



\clearpage

\section{Introdução}

Os sistemas operacionais modernos implementam a maior parte de sua funcionalidade
por meio de extensões carregadas dinamicamente que proveem suporte para dispositivos
de E / S, sistemas de arquivos, redes, etc. Essas extensões são executadas no modo
privilegiado da CPU e, portanto, devem ser verificadas  para assegurar a
ausência de código malicioso. Tradicionalmente a validação destas extensões é
estabelecida através do uso de testes para eliminar bugs. Além disso, as ferramentas de
verificação formal são usadas, em alguns casos, para obter maior segurança~\cite{Ball06,Akash14}.
Apesar de tais ferramentas serem eficazes para encontrar bugs, estas não possuem fortes
garantias de correção.

As extensões do kernel são um tipo especial de aplicação  que se originam de fontes não confiáveis ​​e,
portanto, não podem ser consideradas seguras. Essas extensões ​​permitem que aplicativos personalizem o
kernel com algoritmos específicos para processamento de pacotes, políticas de segurança,
profiling e até modifiquem como os principais subsistemas do kernel~\cite{Nadav17}.

No passado, os sistemas operacionais utilizavam técnicas como sandboxing para a execução de extensões
não confiáveis dentro do kernel. Além disso, algumas abordagens se valiam de linguagens de
domínio específico de domínio~\cite{Bershad95, Fahndrich06} e interpretadores
de bytecode~\cite{McCanne93}. Porém, essas abordagens se mostraram restritivas para alguns casos em que
o desempenho é um fator crítico.

Como uma forma de amenizar essas falhas, o Linux adotou uma abordagem baseada no uso de um
verificador de código, chamado de Exteded Berkeley Package Filters (eBPF). Programas eBPF
consistem de bytecode simples que é compilado em instruções nativas da CPU quando carregado pelo
kernel. Ao contrário dos bytecodes da Java Virtual Machine, o compilador e o run-time de eBPF
não impõe nenhum restrição de tipo. A validade de programas eBPF é imposta por um
verificador estático que impede que programas possam acessar estruturas de dados arbitrárias
do kernel.

Atualmente, programadores eBPF enfrentam quatro grandes problemas. O primeiro deles é que
o verificador de eBPF apresenta uma taxa elevada de falsos positivos, isto é, rejeita como
inseguros programas que não oferecem risco algum ao kernel. Essa limitação obriga desenvolvedores
a reescrever seu código de forma a este ser aceito pelo verificador. Muitas vezes, essas
alterações envolvem a inserção de acessos e verificações redundantes. O segundo problema é que
o algoritmo utilizado pelo verificado é sensível ao número de caminhos presente no programa analisado.
Terceiro, atualmente o verificador impõe a séria restrição de que programas não devem possuir loops.
Finalmente, o verificador atual não é especificado formalmente e isso dificulta a predição de quando
um código será ou não por ele recusado.

Diante do apresentado, o presente projeto pretende especificar formalmente um verificador de
programas eBPF de forma a solucionar as limitações apresentadas. Para isso, utilizaremos técnicas
conhecidas do projeto de linguagens de programação, a saber: semântica formal e
sistemas de tipos~\cite{Pierce02,Klein12,Chang17}.


\section{Revisão da Literatura}

\subsection{A linguagem eBPF}

Aqui você deve apresentar uma introdução à linguagem eBPF.
Explicando o que é a linguagem e o porquê de sua criação.

Adicionalmente, se possível, coloque
exemplos de programas eBPF simples. Como programas eBPF são escritos usando bibliotecas C
ou Python, você pode procurar esses exemplos e explicá-los nessa seção.

\subsection{A linguagem Racket}

Aqui você deve apresentar uma breve introdução à linguagem
racket. Minha idéia é você apresentar sobre a definição de funções,
tipos de dados e macros. Na sequência você pode explicar a implementação daquela
máquina virtual simples que fiz.

\section{Desenvolvimento}

Nesta seção, você deve descrever sua implementação do interpretador de ebPF.


\section{Conclusão e Trabalhos Futuros}

Essa seção vai depender do que você escrever na seção de desenvolvimento. Deixaremos
para escrever por último.


\bibliographystyle{plain}
\bibliography{referencias}
\end{document}
